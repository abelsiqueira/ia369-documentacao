\documentclass[letterpaper,11pt]{article}
\usepackage[brazil]{babel}
\usepackage[T1]{fontenc}
\usepackage{ae}
\usepackage[utf8]{inputenc}
\usepackage[dvipsnames]{color}
\usepackage{graphicx}
\usepackage{epsfig}
\usepackage{amssymb, amsmath, amsfonts}
\usepackage{graphicx}
\bibliographystyle{plain}
\topmargin 0 cm
\hoffset -1 cm
\voffset 0 cm
\evensidemargin 0 cm
\oddsidemargin 0 cm
\setlength{\textwidth}{18 cm}
\setlength{\textheight}{21 cm}

\begin{document}

\begin{enumerate}
\item Acessar \verb+www.github.com+ para criar uma conta para poder utilizá-lo.
\item Ir até ``Signup and Pricing''
\item Clicar em ``Create a Free Account'', preencher os dados solicitados e criar a conta
\item Fazer o download do programa de instalação do Git em 

\verb+http://msysgit.googlecode.com/files/Git-1.7.9-review20120201.exe+
\item Iniciar o aplicativo de instalação e prosseguir até a etapa ``Select Components''.
\item Nesta etapa marcar as opções ``Git Bash Here'' e ``Git Gui Here'' e continuar a instalação
com as opções default.
\item Abrir o Git Bash.
\item Configurar o nome digitando \verb+git config --global user.name "Seu Nome"+
\item Configurar o email digitando \verb+git config --global user.email "Seu Email"+
\item Acessar o repositório criado para a documentação 

\verb+https://github.com/abelsiqueira/ia369-documentacao+ 
e clicar em {\bf fork} para criar sua área no repositório para guardar seus
arquivos
\item Copiar o endereço https gerado para seu repositório
\item No Git, digitar 
\verb+git clone endereco_seu_repositório+: Uma pasta do seu repositório será
criada nos documentos
\item Para enviar um arquivo ao seu repositório, colocá-lo na pasta criada nos documentos
referente a seu repositório e digitar no prompt 
\verb+git add nome_arquivo+
\item Para dar o commit nos arquivos digitar 
\verb+git commit -am "Descrição do arquivo"+
\item Para enviar o arquivo ao repositório, digitar 
\verb+git push origin master+
\item Para avisar o repositório central que existem mudanças no seu
repositório, volte ao endereço do repositório central e clique em
{\bf Pull Request}
\item Para adicionar o repositório central digitar 

{\small
\verb+git remote add central https://github.com/abelsiqueira/ia369-documentacao.git+}
\item Para pegar os arquivos desse diretório digitar no prompt 
\verb+git fetch central+ e
depois 
\verb+git merge central/master+. Neste ponto o Git irá fazer o download de
todos os arquivos que estiverem lá.
\item Para fazer o envio de alguma alteração que for feita nos arquivos
 usar o 
\verb+git commit -am "Descrição das mudanças"+
\item Para enviar as mudanças ao repositório, digitar 
\verb+git push origin master+
\item Para avisar o repositório central que existem mudanças no seu
repositório, volte ao endereço do repositório central e clique em
{\bf Pull Request}
\end{enumerate}
\end{document}

