\documentclass[letterpaper,11pt]{article}
\usepackage[brazil]{babel}
\usepackage[T1]{fontenc}
\usepackage{ae}
\usepackage[utf8]{inputenc}
\usepackage[dvipsnames]{color}
\usepackage{graphicx}
\usepackage{epsfig}
\usepackage{amssymb, amsmath, amsfonts}
\usepackage{graphicx}
\bibliographystyle{plain}
\topmargin 0 cm
\hoffset -1 cm
\voffset 0 cm
\evensidemargin 0 cm
\oddsidemargin 0 cm
\setlength{\textwidth}{18 cm}
\setlength{\textheight}{21 cm}

\title{ Exemplo de arquivo .tex }
\author{ Abel Siqueira }
\date{}

\begin{document}

\maketitle

\section{Seção}
\subsection{Subseção}
\subsubsection{Subsubseção}

A quebra de linha
no latex não quebra a linha
no pdf.

Dar um linha de espaço cria um parágrafo.

Lista com bullets.
\begin{itemize}
 \item Item 1
 \item Item 2
 \begin {itemize}
  \item Item 2.1
  \item Item 2.2
 \end{itemize}
\end{itemize}

Lista numerada
\begin{enumerate}
 \item Item 1
 \item Item 2
 \begin{itemize}
  \item Item 2.1
  \item Item 2.2
 \end{itemize}
\end{enumerate}

Incluir figura
\begin{flushleft}
 Esquerda
 \includegraphics[scale=0.25]{logo.png}
 \includegraphics[scale=0.5]{logo.png}
 \includegraphics[scale=1.0]{logo.png}
 \includegraphics[scale=2.0]{logo.png}
\end{flushleft}
\begin{center}
 Centro
 \includegraphics[scale=0.25]{logo.png}
 \includegraphics[scale=0.5]{logo.png}
 \includegraphics[scale=1.0]{logo.png}
 \includegraphics[scale=2.0]{logo.png}
\end{center}
\begin{flushright}
 Direita
 \includegraphics[scale=0.25]{logo.png}
 \includegraphics[scale=0.5]{logo.png}
 \includegraphics[scale=1.0]{logo.png}
 \includegraphics[scale=2.0]{logo.png}
\end{flushright}

Alguns caracteres não podem ser usados no texto, como \verb+_+, \verb+\+, \verb+^+. Para usá-los, utilizamos o comando \verb-\verb+_+-, onde os + são delimitadores e tudo dentro é aceito sem formatação. Para textos existe o
ambiente verbatim:
\begin{verbatim}
Aqui dentro
  não existe
    formatação
_^~'"\
\verb
\end{verbatim}

\end{document}

