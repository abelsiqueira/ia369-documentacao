\section{Interface}

Nesta seção, descrevemos os elementos de interação entre o jogador e o
jogo.

\subsection{HUD}

A tela do jogador terá as informações a seguir, seguindo o modelo da
Figura \ref{fig:hud}
\begin{figure}[!ht]
 \centering
 \includegraphics[scale=0.59]{hud.png}
 \caption{HUD}
 \label{fig:hud}
\end{figure}
\begin{enumerate}
 \item {\bf Marcador de Vida:} Canto superior esquerdo da tela, no 
formato de uma faixa circular. Na figura está demonstrado pela faixa
verde.
 \item {\bf Marcador de Energia/Temperatura:} Quando houver este marcador,
ele estará no canto superior esquerdo da tela, também no formato
de uma faixa circular, dentro da faixa do marcador de vida. Na primeira
fase, será marcador de energia, e na segunda fase de temperatura. Não
existirá esse marcador na terceira fase.
Na figura está indicado pela faixa azul dentro da faixa verde. 
 \item {\bf Minimapa:} Canto inferior direito, em forma de quadrado.
Será fixo e semi-transparente. Mostrará um esboço do mapa e um triângulo
indicando a posição e direção de Medrash. Existirá apenas na primeira fase.
 \item {\bf Barra de Vida dos inimigos:} Ficará sobre a cabeça do inimigo,
indicando quantos pontos de vida aquele inimigo tem. Indicado na figura
como barras vermelhas.
 \item {\bf Barra de tempo:} Esta barra indicará, que existe apenas na 
fase três, indicará o tempo restante para completar a batalha final.
Seu tamanho irá diminuir com o tempo, indicando o tempo que passa. Na 
figura, a barra azul no topo da tela indica a Barra de Tempo.
\end{enumerate}

\subsection{Menus}

O jogo terá três menus. Um deles será o menu principal do jogo, o 
outro será o menu de pausa, e o último será um menu entre fases.

\subsubsection{Menu Principal}

O menu principal terá as seguintes opções:
\begin{itemize}
 \item {\bf Começar novo jogo:} Apaga todas informações sobre o jogo atual,
se houver, e inicia um novo jogo.
 \item {\bf Continuar jogo:} Se houver um jogo em progresso, permite que
o usuário continue o mesmo. Caso contrário, esta opção estará desativada.
 \item {\bf Carregar jogo:} Permite que o usuário carregue algum jogo
gravado, se houver.
 \item {\bf Sair do jogo:} Sai do jogo, finalizando o software.
\end{itemize}

\subsubsection{Menu de Pausa}
O menu de pausa só pode ser acessado durante o jogo, e contém as seguintes
opções:
\begin{itemize}
 \item {\bf Continuar o jogo:} Sai do menu de pausa.
 \item {\bf Voltar ao menu principal:} Sai do jogo e volta ao menu 
principal. Não salva o jogo.
 \item {\bf Sair do jogo:} Sai do jogo e finaliza o software.
\end{itemize}

\subsection{Menu entre fases}
Ao fim das fases 1 e 2 será exibido um menu que irá mostrar a pontuação do
personagem, assim como as opções:
\begin{itemize}
 \item {\bf Salvar jogo:} Permite que o jogador salve o progresso.
 \item {\bf Próxima fase:} Continua para a próxima fase.
 \item {\bf Voltar ao menu principal:} Sai do jogo e volta ao menu principal.
 \item {\bf Sair do jogo:} Sai do jogo e finaliza o software.
\end{itemize}
