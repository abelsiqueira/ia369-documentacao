\section{Mecânica do Jogo}

Nesta sessão serão descritas as regras de interação entre o jogador e o jogo.

\subsection {Mecânica Básica}

O jogador irá passar por mapas do jogo tendo a liberdade de se deslocar em
 qualquer direção dentro da área permitida.

\subsubsection {Controles}

O jogador poderá movimentar o personagem pelo cenário por meio das setas
 direcionais. Para as demais interações serão utilizadas as teclas
 ``A'', ``S'', ``D'' e a barra de espaço:
\begin{itemize}
\item {\bf ``A'' é o botão de ataque} O personagem utiliza a arma disponível
para atacar o inimigo mais próximo;
\item {\bf ``S'' é o botão de ação} O personagem realiza a interação com o
 cenário e outros personagens;
\item {\bf ``D'' é o botão de defesa} Quando não for possível sair da frente
 de um ataque, a tecla ``D'' pode ser usada para se defender;
\item {\bf A barra de espaço é o botão de pulo} O personagem pula.
\end{itemize}

\subsubsection {Deslocamento}
Na primeira fase o personagem principal se desloca inicialmente correndo. 
Conforme este perde energia, passa a correr mais devagar até que atinja um 
nível crítico onde passa apenas a andar. Nas demais, não haverá indicador 
de canseira.

\subsubsection {Inimigos e Desafios}
Nos mapas irão existir diversos inimigos que podem detectar o jogador caso 
este chegue muito próximo. Sendo verdade, eles iniciam uma perseguição. Assim 
que o jogador se afasta a uma determinada distância, os inimigos voltam ao 
seu estado inicial. 

Os inimigos quando próximos do jogador irão atacá-lo, possibilitando que o 
personagem também faça o mesmo. 
Adicionalmente, no final de cada mapa haverá um desafio extra, como um inimigo
 mais forte ou uma batalha.  

\subsubsection {Interação com o Cenário}

Além dos inimigos, o personagem principal poderá interagir com partes do
 cenário e NPCs para completar os desafios propostos.

\subsection {Combate}
\subsubsection{Inimigos}
Ao se aproximar dos inimigos, estes irão atacar o personagem principal. A
tecla ``A'' pode então ser pressionada para atacar o mais próximo.

Os ataques dos inimigos serão periódicos, podendo ser defendidos utilizando a
 tecla ``D'', que irá reduzir o dano causado. Também podem ser esquivados
 utilizando as setas direcionais para sair da região de efeito do ataque.

\subsection {Dificuldade}
O jogo apresentará duas classes de dificuldades ao personagem principal: de percurso e de
inimigos. As dificuldades de percurso serão aquelas apresentadas pelo ambiente e que exigirão a habilidade
de locomoção do personagem, tais como salto ou desvio de obstáculos. As dificuldades de inimigos serão
aquelas que exigirão a habilidade de ataque e defesa do personagem quando em confronto com um inimigo.

De uma fase para outra serão introduzidas novas mecânicas de jogo aumentando o
grau de dificuldade e criando um incentivo para que o jogador não perca o interesse pelo jogo.

\subsection {Vida, Energia e Temperatura}

Existem três barras principais durante o jogo: as barras de vida, energia e temperatura. 

A barra de vida indica a saúde do personagem. Esta barra diminui quando
o personagem sofre ataques dos inimigos e aumenta quando Medrash come os alimentos
encontrados.

A barra de energia representa um desafio extra durante a primeira fase. Esta
 barra diminui conforme o deslocamento do personagem e aumenta quando o 
mesmo cumpre algumas necessidades, tais como comer.

A barra de temperatura existe apenas na segunda fase e serve para mostrar que
o personagem está perdendo temperatura corporal. Esta barra aumenta quando o 
personagem aproxima-se do fogo.

\subsection {Salvar e Carregar o jogo}
Para salvar o progresso no jogo será utilizado um sistema de perfil. Inicialmente o jogador criará um perfil e este dará acesso à primeira fase. Após a conclusão das fases, as posteriores serão devidamente liberadas, com a opção de salvar o progresso obtido liberada. Somente é possível salvar um jogo no início de uma nova fase.

Será possível criar diversos perfis. Na tela inicial do jogo haverá uma opção de carregar o perfil e esta irá conter todos os existentes.