\section{Mecânica do Jogo}


Nesta sessão estão descritas as regras de interação entre o jogador e o jogo.

\subsection {Mecânica Basica}

O jogador irá passar por mapas do jogo tendo a liberdade de se deslocar em
 qualquer direção até que encontre um \textit{checkpoint} que estará em uma região
 do mapa que não permite a volta do personagem a área anterior.

\subsubsection {Controles}

O jogador poderá movimentar o personagem pelo cenário por meio das setas
 direcionais. Para as demais interações serão utilizadas as teclas
 ``A'', ``S'', ``D'' e a barra de espaço.
\begin{itemize}
\item {\bf ``A'' é o botão de ataque} O personagem utiliza a arma disponível
para atacar o inimigo mais próximo.
\item {\bf ``S'' é o botão de ação} O personagem realiza a interação com o
 cenário e outros personagens.
\item {\bf ``D'' é o botão de defesa} Quando não for possível sair da frente
 de um ataque a tecla ``D'' pode ser usada para se defender.
\item {\bf A barra de espaço é o botão de pulo} O personagem pula.
\end{itemize}

\subsubsection {Deslocamento}

Na primeira fase o personagem principal se desloca inicialmente correndo. Conforme este 
perde energia passa a correr mais devagar até que atinja um nível crítico
onde o personagem passa apenas a andar.

Nas demais fases o personagem tem apenas a opção de correr.

\subsubsection {Inimigos e Desafios}

Nos mapas vão existir diversos inimigos que podem detectar o jogador se 
este chegar muito próximo 
e então dão inicio a uma perseguição, assim que o jogador se afasta a 
uma determinada distância esses voltam ao seu estado inicial. 

Estes inimigos quando próximos do jogador irão ataca-lo possibilitando que o 
jogador também faça o mesmo. 
Adicionalmente no final de cada mapa haverá um desafio extra como um inimigo
 mais forte ou uma batalha.  

\subsubsection {Interação com o Cenário}

Além dos inimigos o personagem principal poderá interagir com partes do
 cenário e outros personagens para completar os
desafios propostos.

\subsection {Combate}
\subsubsection{Inimigos}
Ao se aproximar dos inimigos, estes irão atacar o personagem principal. O
 personagem principal pode então pressionar a tecla ``A'' para atacar o
 inimigo mais próximo.

Os ataques dos inimigos serão periódicos podendo ser defendidos utilizando a
 tecla ``D'', que irá reduzir o dano causado, ou podem ser esquivados
 utilizando as 
setas direcionais para sair da região de efeito do ataque.

\subsubsection {Armas}
O personagem principal começa armado com um porrete de madeira. Durante a
 primeira fase este pode pegar uma lança que irá facilitar no combate com o
 tigre. 

Na segunda fase haverá um bastão em chamas que, entre outros usos, espanta
 os animais que estão próximos.

\subsection {Dificuldade}

O jogo apresentará duas classes de dificuldades ao personagem principal (ou jogador): de percurso e de
inimigos. As dificuldades de percurso serão aquelas apresentadas pelo ambiente e que exigirá a habilidade
de locomoção do personagem, tal como salto ou desvio de obstáculos. As dificuldades de inimigos serão
aquelas que exigirão a habilidade de ataque e defesa do personagem quando em confronto com um inimigo.

De uma fase para outra são introduzidas novas mecânicas de jogo (ver capitulo 4) aumentando o
grau de dificuldade e criando um incentivo para que o jogador não perca o interesse pelo jogo.

\subsection {Vida, Energia e Temperatura}

Existem três barras principais durante o jogo, as barras de vida, energia e temperatura. 

A barra de vida indica a saúde do personagem, esta barra diminui quando
o personagem sofre ataques dos inimigos e aumenta quando esse derrota os
 inimigos.

A barra de energia representa um desafio extra durante a primeira fase, esta
 barra diminui conforme o deslocamento do personagem e aumenta quando o 
mesmo cumpre alguns objetivos na fase como matar animais.

A barra de temperatura existe apenas na segunda fase e serve para mostrar que
o personagem está perdendo temperatura corporal e está aumenta quando o personagem está perto do fogo.

\subsection {Salvar e Carregar o jogo}
Para salvar o progresso no jogo será utilizado um sitema de perfil. Inicialmente o jogador criará um perfil e este dará acesso 
a primeira fase, quando o jogador passar pela primeira fase este perfil dará acesso a segunda fase e será registrado com qual arma o jogador
finalizou a primeira etapa. Ao passar pela segunda fase o jogador então terá acesso à terceira fase no perfil. O jogo apenas salva o progresso 
no inicio de cada fase.

Será possivel criar diversos perfis. Na tela inicial do jogo haverá uma opção de carregar o jogo e está listará todos os perfis existentes.