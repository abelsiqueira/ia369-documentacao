\section{Testes}
Com o decorrer dos anos, aplicações de software robustas e complexas
 demandam cada vez mais esforços para que se mantenha sua qualidade. A
 engenharia de software é constituída por um grupo de métodos, ferramentas e
 critérios cujo objetivo recai sobre a tentativa de produzir aplicativos
 detentores desta qualidade. Dentre os processos propostos por ela, estão
 aqueles responsáveis por descrever as atividades de teste de software.
   
O teste de software refere-se a um processo de âmbito investigativo, ou
 seja, as ferramentas e métodos propostos  por ele têm por objetivo procurar
 e encontrar problemas presentes em  uma aplicação. Sendo verdade, pode-se
 dizer que trata-se de uma investigação a respeito da qualidade do
 aplicativo dentro do contexto ao qual ele deve operar. Segundo
 \cite{bib:pressman} o objetivo central do teste é o de apontar a existência
 da maior quantidade possível de defeitos que  não foram identificados pelas
 revisões dentro dos limites de prazo e de custo.

Diz-se que um caso de teste é bom quando apresenta grande probabilidade de
 revelar a presença de um defeito ainda não descoberto. Portanto, pode-se
 dizer que um teste é bem-sucedido quando aponta e verifica a presença de um
 defeito de maneira satisfatória.
 
Um ponto de vista errôneo geralmente extraído de um aplicativo
 disponibilizado para ser comercializado é de que ele deve funcionar
 corretamente, sem a presença de erros. A complexidade envolvida em sua
 codificação, somada aos profissionais presentes, torna impraticável a
 possibilidade de provar esta corretitude.
 
As falhas de aplicativos podem ser geradas por uma grande gama de motivos,
 dentre os quais, especificações de requisitos erradas, implementações
 incorretas e plataformas de hardware não suportáveis. 

No âmbito do teste de software há uma distinção clara entre os conceitos de
 defeito, erro e falha, cada qual com suas características e
 particularidades. Uma falha é uma condição anormal na saída que era
 esperada por um componente do software, ou seja, algo que é visível ao
 usuário. O defeito é algo causador da falha e, portanto, pode-se dizer que
 se caracteriza por ser algum problema com o código do software. Por fim, o
 erro é um estado inconsistente que se propaga pelo sistema após o exercício
 de um defeito. Sendo verdade, diz-se que o erro pode ou não acarretar em
 uma falha devido a fatores tais como redundância de processamento.

Pressman discorre sobre a atividade de teste da seguinte maneira: “Se
 realmente fossemos bons para programar não haveria bugs. Se existem bugs, é
 porque somos ruins naquilo que fazemos e, se somos ruins nisso devemos
 sentir-nos culpados por isso. Assim, a atividade de teste e o projeto de
 casos de teste são 
uma admissão de falha, o que promove uma boa dose de culpa. O tédio de
 testar é apenas uma punição por nossos erros”. 

As principais considerações do teste de software são: 

\begin{itemize}
\item Para ser eficaz o teste deve ser cuidadosamente desenhado;
\item Testes improvisados devem ser evitados;
\item Resultados devem ser inspecionados e comparados com resultados esperados;
\item Desenvolvedores não são as pessoas mais indicadas para testar seu próprio produto;
\item Testadores independentes são importantes;
\end{itemize}

No que diz respeito às abordagens para o teste de software pode-se dizer que
 há três: teste de caixa branca, de caixa preta e caixa cinza.

O teste caixa preta, do inglês \textit{Black Box}, ou teste funcional,
 caracteriza-se por não considerar o código fonte da aplicação ao efetuar os
 testes. Responsabiliza-se por determinar se os requisitos foram total ou
 parcialmente satisfeitos com a verificação dos resultados obtidos, deixando
 de lado o modo como ocorreu o processamento. Também demonstra quais são as
 funções do software que encontram-se operacionais, ou seja, aquelas que
 produzem a saída esperada. Dentre as principais técnicas que utilizam-se
 desta abordagem cita-se o particionamento de equivalência e a análise do
 valor limite.

O teste caixa branca, do inglês \textit{White Box}, ou teste estrutural, de
 maneira oposta ao que ocorre com o caixa preta, considera o código da
 aplicação na aplicação das atividades de teste. Responsabiliza-se por
 determinar defeitos na estrutura interna do programa por meio do exercício
 dos caminhos de execução, com a execução de conjuntos específicos de
 condições ou laços. Dentre as principais técnicas
 que utilizam-se desta abordagem cita-se o teste de caminho básico, o qual
 utiliza um grafo de programa para denotar as instruções presentes no código
 fonte.

O teste caixa cinza, do inglês \textit{Gray Box}, caracteriza-se por ser um
 híbrido entre os dois anteriormente descritos, caixa branca e preta, por
 apresentar elementos e características presentes em ambos.

Para o jogo ``As Crônicas de Medrash'' a ser desenvolvido, testes funcionais
 serão realizados em detrimento dos estruturais. O jogo será testado de
 acordo com as especificações presentes no documento de \textit{Game
 Design}. Serão criado casos de testes para exercitar aspectos variados
 referentes ao jogo tais como fluxo das fases, inteligência artificial do
 inimigos e comportamentos do personagem principal.
 
A criação e execução dos casos de testes deverá ser feita por pessoal
 especializado. A correção dos eventuais problemas encontrados será cargo do
 desenvolvedor responsável. Poderá também haver o re-teste no âmbito de se
 verificar se o problema foi realmente corrigido ou se não gerou novos com
 a solução aplicada.
